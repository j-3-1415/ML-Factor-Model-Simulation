%-------------------------------------------------------------------------------
%	PACKAGES AND OTHER DOCUMENT CONFIGURATIONS
%-------------------------------------------------------------------------------


\documentclass[a4paper,12pt]{article}
\usepackage[english]{babel}
% \usepackage[latin1]{inputenc}
\usepackage{amsmath}
\usepackage{amssymb}
\usepackage{amsfonts}
\usepackage{graphicx}
\usepackage{dcolumn}
\usepackage[colorinlistoftodos]{todonotes}
\usepackage[toc,page]{appendix}
\usepackage{setspace}
% \doublespacing
\usepackage{tcolorbox}
\tcbuselibrary{skins}
\usepackage{booktabs}
\usepackage{hyperref}
\usepackage{geometry}
\usepackage[bottom]{footmisc}
\usepackage{listings}
\usepackage{longtable}
% \usepackage[demo]{graphicx}
\usepackage{subfig}
\usepackage{multirow}
\usepackage{tikz}
\usetikzlibrary{fit}
\usetikzlibrary{arrows}
\renewcommand{\arraystretch}{0.7}
\renewcommand{\labelitemi}{$\triangleright$}
 \geometry{
 a4paper,
 total={170mm,257mm},
 left=25mm,
 top=30mm,
 right=25mm,
 bottom=25mm,
 }
 \usepackage{hyperref}
 \hypersetup{
    bookmarks=true,         % show bookmarks bar?
    unicode=false,          % non-Latin characters in Acrobat’s bookmarks
    pdftoolbar=true,        % show Acrobat’s toolbar?
    pdfmenubar=true,        % show Acrobat’s menu?
    pdffitwindow=false,     % window fit to page when opened
    pdfstartview={FitH},    % fits the width of the page to the window
    pdftitle={Design_Report},    % title
    pdfauthor={Jacob Pichelman, Luca Poll},     % author
    pdfsubject={Subject},   % subject of the document
    pdfcreator={Creator},   % creator of the document
    pdfproducer={Producer}, % producer of the document
    pdfkeywords={keyword1, key2, key3}, % list of keywords
    pdfnewwindow=true,      % links in new PDF window
    colorlinks=false,       % false: boxed links; true: colored links
    linkcolor=red,          % color of internal links (change box color with linkbordercolor)
    citecolor=green,        % color of links to bibliography
    filecolor=magenta,      % color of file links
    urlcolor=cyan           % color of external links
}

\definecolor{Gray}{gray}{0.9}
% \setmonofont{Consolas}

\definecolor{background}{RGB}{39, 40, 34}
\definecolor{string}{RGB}{230, 219, 116}
\definecolor{comment}{RGB}{117, 113, 94}
\definecolor{normal}{RGB}{248, 248, 242}
\definecolor{identifier}{RGB}{166, 226, 46}

\lstset{
  language = R,                         % choose the language of the code
  linewidth = 14.75cm,
  numbers = left,                           % where to put the line-numbers
  stepnumber=1,                         % the step between two line-numbers.        
  numbersep=5pt,                        % how far the line-numbers are from the code
  numberstyle=\tiny\color{black}\ttfamily,
  backgroundcolor=\color{background},       % choose the background color. You must add \usepackage{color}
  showspaces=false,                     % show spaces adding particular underscores
  showstringspaces=false,               % underline spaces within strings
  showtabs=false,                       % show tabs within strings adding particular underscores
  tabsize=4,                            % sets default tabsize to 2 spaces
  captionpos=b,                         % sets the caption-position to bottom
  breaklines=true,                      % sets automatic line breaking
  breakatwhitespace=true,               % sets if automatic breaks should only happen at whitespace
  title=\lstname,                       % show the filename of files included with \lstinputlisting;
  basicstyle=\color{normal}\ttfamily,                   % sets font style for the code
  keywordstyle=\color{magenta}\ttfamily,    % sets color for keywords
  stringstyle=\color{string}\ttfamily,      % sets color for strings
  commentstyle=\color{comment}\ttfamily,    % sets color for comments
  emph={format_string, eff_ana_bf, permute, eff_ana_btr},
  emphstyle=\color{identifier}\ttfamily
}

% biblatex
\usepackage[
backend=biber,
style=apa,
citestyle = authoryear
]{biblatex}

%\usepackage{biblatex}
\bibliography{References}
% \usepackage{csquotes}


\begin{document}
\begin{titlepage}

\newcommand{\HRule}{\rule{\linewidth}{0.25mm}} % Defines a new command for the horizontal lines, change thickness here
\setlength{\topmargin}{-0.5in}
\center % Center everything on the page

\includegraphics[scale=0.75]{TSE.png}\\

%-------------------------------------------------------------------------------
%	HEADING SECTIONS
%-------------------------------------------------------------------------------
% \\[1.5cm]
\large \textsc{M2 EEE Machine Learning} 
\vspace{1.5cm}
% Name of your heading such as course name
\textsc{\large } % Minor heading such as course title

%-------------------------------------------------------------------------------
%	TITLE SECTION
%-------------------------------------------------------------------------------

\HRule \\[0.75cm]
{ \huge \bfseries Final Project}\\[0.5cm] % Title of your document
\HRule \\[1.75cm]
 
%-------------------------------------------------------------------------------
%	AUTHOR SECTION
%-------------------------------------------------------------------------------

\large\textsc{Andrew Boomer and Jacob Pichelmann} \\[1.5cm]

%-------------------------------------------------------------------------------
%	DATE SECTION
%-------------------------------------------------------------------------------

{\large \today}\\[0.5cm] % Date, change the \today to a set date if you want to be precise

\vfill % Fill the rest of the page with whitespace

\end{titlepage}

% -------------------------------------------------------------
% TABLE OF CONTENTS
\renewcommand{\contentsname}{Table of Contents}
\tableofcontents
% -------------------------------------------------------------

\newpage

\section{Introduction}

Four dimension reduction devices: (1) Principal Component Analysis, (2) Ridge Regression, (3) Landweber Fridman (LF) regularization, (4) Partial least squares. Each involves a regularization or tuning parameter that is selected through genearlized cross validation (GCV). 

Take two different data generating processes, (1) The eigenvalues of $\frac{X^{'}X}{T}$ are bounded and decline to zero gradually. (2) Popular factor model with a finite number, r, of factors. Here, the r largest eigenvalues grow with N, while the remaining are bounded. In both cases, $\frac{X^{'}X}{T}$ is ill-conditioned, which means the ratio of the largest to smallest eigenvalue diverges, and a regularization terms is needed to invert the matrix.

\section{Data Generating Process}

Large Sample Case: $N = 200$ and $T = 500$
Small Sample Case: $N = 100$ and $T = 50$

\[\underbrace{x_{t}}_{(N \times 1)} = \underbrace{\Lambda}_{(N \times r)} \underbrace{F_{t}}_{(r \times 1)} + \underbrace{\xi_{t}}_{(N \times 1)}\]

\[\underbrace{y_{t}}_{(1 \times 1)} = \underbrace{\theta^{'}}_{(1 \times r)} \underbrace{F_{t}}_{(r \times 1)} + \underbrace{\nu_{t}}_{(1 \times 1)}\]

\[\underbrace{y}_{(T \times 1)} = \underbrace{F}_{(T \times r)} \underbrace{\theta}_{(r \times 1)} + \underbrace{\nu}_{(T \times 1)}\]

\[\underbrace{X}_{(T \times N)} = \underbrace{F}_{(T \times r)} \underbrace{\Lambda^{'}}_{(r \times N)} + \underbrace{\xi}_{(T \times N)}\]

DGP 1 (Few Factors Structure):
$\theta$ is the $(r \times 1)$ vector of ones, $r = 4$ and $r_{max} = r + 10$

DGP 2 (Many Factors Structure):
$\theta$ is the $(r \times 1)$ vector of ones, $r = 50$ and $r_{max} = min(N, \frac{T}{2})$

DGP 3 (Five Factors but only One Relevant):
$\theta = (1, 0_{1 \times 4})$, $r = 5$ and $r_{max} = min(r + 10, min(N, \frac{T}{2}))$

$F = [F_{1}, F_{2}^{'}]^{'}$ and $F \times F^{'} = \begin{bmatrix} 1 & 0 & 0 & 0 & 0 \\ 0 & 2 & 0 & 0 & 0 \\ 0 & 0 & 3 & 0 & 0 \\ 0 & 0 & 0 & 3 & 0 \\ 0 & 0 & 0 & 0 & 4\end{bmatrix}$

$y = \hat{F} \theta + \nu$ where $\hat{F}$ is generated from $X$ equation in DGP 3, and $\sigma_{\nu} = 0.1$

DGP 4 ($x_{t}$ Has a Factor Structure but Unrelated to $y_{t}$):

$\theta$ is a vector of zeros with dimension $(r \times 1)$. $r = 5$, $r_{max} = r + 10$. $F \times F^{'}$ is defined as in DGP 3.

DGP 5 (Eigenvalues Declining Slowly):

$\theta$ is an $(N \times 1)$ vector of ones. $r = N$, $r_{max} = min(N, \frac{T}{2}$.

$\Lambda = M \odot \xi$, with $\xi \sim (N \times N)$ matrix of $iidN(0, 1)$

$M \sim (N \times N) = \begin{bmatrix} 1 & 1 & \dotsb & 1 \\ \frac{1}{2} & \frac{1}{2} & \dotsb & \frac{1}{2} \\ \vdots & \vdots & \vdots & \vdots \\ \frac{1}{N} & \frac{1}{N} & \dotsb & \frac{1}{N}\end{bmatrix}$

DGP 6 (Near Factor Model):

$\theta = 1$, $r = 1$, $r_{max} = r + 10$, $\Lambda^{'} = \frac{1}{\sqrt{N}}1_{r \times N}$


Estimation:

Set 1: Bai-Ng, PCA, PLS, Ridge, LF, LASSO
Set 2: GCV, Mallows, AIC, BIC
Set 3: Small Sample, Large Sample

Parameter Iteration (For Later)
Simulation (Andy)
Model Estimation (Jacob)
Evaluation (Jacob)
Output (Andy)

\section{Estimation Methods}

Ridge Estimator:

\[\widehat{y} = M_{T}^{\alpha} y = X (S_{xx} + \alpha I)^{-1} S_{xy}\]

LF Estimator:

\[\widehat{y} = M_{T}^{\alpha} y = X \sum_{j=1}^{\min (N, T)} \frac{\left(1-\left(1-d \widehat{\lambda}_{j}^{2}\right)^{1 / \alpha}\right)}{\widehat{\lambda}_{j}^{2}}\left\langle y, \hat{\psi}_{j}\right\rangle_{T} \frac{X^{\prime} \hat{\psi}_{j}}{T}\]

Spectral Cutoff/Principal Components Estimator:

\[\widehat{y} = M_{T}^{\alpha} y = \widehat{\Psi} \left(\widehat{\Psi}^{\prime} \widehat{\Psi}\right)^{-1} \widehat{\Psi}^{\prime} y\]

\[\text{ Where } \widehat{\Psi} = \left[\widehat{\psi}_{1}\left|\widehat{\psi}_{2}\right| \ldots \mid \widehat{\psi}_{k}\right]\]

Partial Least Squares Estimator:

\[\widehat{y} = M_{T}^{\alpha} y = X V_{k}\left(V_{k}^{\prime} X^{\prime} X V_{k}\right)^{-1} V_{k}^{\prime} X^{\prime} y\]

\[\text{ Where } V_{k}=\left(X^{\prime} y, \quad\left(X^{\prime} X\right) X^{\prime} y, \ldots,\left(X^{\prime} X\right)^{k-1} X^{\prime} y\right)\]

\subsection{Via SIMPLS Algorithm}

\begin{align}
\nonumber S = X^{T} y & \\
\nonumber \text{for } & i \in 1:k \\
\nonumber &\text{if } i = 1, [u, s, v] = svd(S) \\
\nonumber &\text{if } i > 1, [u, s, v] = svd(S - (P_{k}[:, i-1](P_{k}[:, i-1]^{T} P_{k}[:, i-1])^{-1} P_{k}[:, i-1]^{T} S)) \\
\nonumber &T_{k}[:, i - 1] = X R_{k}[:, i - 1] \\
\nonumber &P_{k}[:, i - 1] = \frac{X^{T} T_{k}[:, i - 1]}{T_{k}[:, i - 1]^{T}T_{k}[:, i - 1]} \\
\nonumber \widehat{y} = M^{\alpha}_{T} y &= X R_{k} (T^{T}_{k} T_{k})^{-1} T^{T}_{k} y
\end{align}

\section{Evaluation Methods}

Generalized Cross Validation:

\[\hat{\alpha}=\arg \min _{\alpha \in A_{T}} \frac{T^{-1}\left\|y-M_{T}^{\alpha} y\right\|^{2}}{\left(1-T^{-1} \operatorname{tr}\left(M_{T}^{\alpha}\right)\right)^{2}}\]

Mallows Criterion:

\[\hat{\alpha}=\arg \min _{\alpha \in A_{T}} T^{-1}\left\|y-M_{T}^{\alpha} y\right\|^{2}+2 \widehat{\sigma}_{\varepsilon}^{2} T^{-1} \operatorname{tr}\left(M_{T}^{\alpha}\right)\]

\[\text{ Where } \widehat{\sigma}_{\epsilon}^{2} \text{ is a consistent estimator of the variance of } \epsilon\]

So variance of $\epsilon$ is taken from the errors of the largest model, or from the model with all regressors for PC.

Leave-one-out Cross Validation:

\[\hat{\alpha}=\arg \min _{\alpha \in A_{T}} \frac{1}{T} \sum_{t=1}^{T}\left(\frac{y_{i}-\hat{y}_{i, \alpha}}{1-M_{T}^{\alpha}[ii]}\right)^{2}\]

\clearpage

\begin{center}
	Table 1. Panel A $(\mathrm{GCV}, N=200, T=500)$ \\[-1.8ex]
\end{center}
\begin{tabular}{ccccccccccc}
\hline \hline\\[-1.8ex] & & Bai-Ng & \multicolumn{4}{c} { PC with cross-validation } & \multicolumn{3}{c} { PLS } \\
\hline & $r$ & $k$ & $k$ & MSER & \multicolumn{2}{c} { RMSFER } & $k$ & MSER & \multicolumn{2}{c} { RMSFER } \\
\hline & & & & & rec & roll & & & rec & roll \\
\hline DGP 1 & 4.00 & 4.00 & 4.83 & 1.00 & 1.01 & 1.02 & 3.16 & 0.93 & 1.05 & 1.06 \\
(s.e.) & $-$ & 0.00 & 1.90 & 1.00 & 1.33 & 1.74 & 0.40 & 1.45 & 1.28 & 1.35 \\
DGP 2 & 50.00 & 200.00 & 50.96 & 1.50 & 0.49 & 0.25 & 8.99 & 1.44 & 0.48 & 0.25 \\
(s.e.) & $-$ & 0.00 & 2.59 & 1.35 & 0.68 & 0.32 & 0.61 & 1.26 & 0.48 & 0.19 \\
DGP 3 & 5.00 & 5.61 & 1.93 & 1.00 & 1.00 & 0.99 & 2.12 & 0.99 & 1.01 & 1.01 \\
(s.e.) & $-$ & 0.71 & 2.11 & 1.02 & 0.99 & 1.00 & 0.47 & 1.08 & 1.06 & 1.10 \\
DGP 4 & 5.00 & 5.61 & 0.86 & 1.01 & 0.99 & 0.98 & 1.36 & 0.99 & 1.01 & 1.01 \\
(s.e.) & $-$ & 0.75 & 1.90 & 1.02 & 1.00 & 0.99 & 0.89 & 1.20 & 1.12 & 1.18 \\
DGP 5 & 200.00 & 200.00 & 11.94 & 1.70 & 0.45 & 0.21 & 2.38 & 1.57 & 0.46 & 0.22 \\
(s.e.) & $-$ & 0.00 & 8.27 & 1.54 & 0.40 & 0.16 & 0.80 & 2.12 & 0.44 & 0.20 \\
DGP 6 & 1.00 & 0.00 & 6.51 & 0.81 & 0.86 & 0.89 & 2.00 & 0.50 & 1.04 & 1.09 \\
(s.e.) & $-$ & 0.00 & 3.38 & 0.84 & 0.87 & 0.91 & 0.00 & 0.60 & 1.23 & 1.37 \\
\hline
\end{tabular}

\end{document}




